\section{CI Server}
Inden da vi skulle Unit teste vores kode fik vi sat vore jenkins server op. Det var ikke et problem da der var en guide på blackboard på hvordan vi satte coverage templaten op. Problemet opstod først da jenkins ikke understøttede .NETCORE 3.0 frameworket. Vi prøvede på at nedgrader til .NETCORE 2.2 det hjalp lidt på vores problem. Derefter manglede vi mange packages unødvendige packages. Da vi fik installeret alle packages løb vi ind i et andet problem. Nu opstod der et fejl med NUnit.Engine. Vi prøvede på at google løsninger til dette problem men vi fandt aldrig et svar til det. Så vi besluttede os for at ændre til .NET framework da jenkins serveren var sat op til .NET frameworket. Nu opstod der et nyt mystisk problem. vi prøvede på at bygge på jenkins, resultatet sagde at alt var rigtigt men kasserne var markeret svagt rødt. Vi kiggede efter feedback fra jenkins men den sagde alt var rigtigt. vi fandt så frem til at vi skulle flytte alle vores visual studio filer til roden af vores repository. dette fixede problemet og nu var jenkins serveren sat op.\\
\medskip
En ting Jenkins serveren har hjulpet os med er at den viser hvor mange procent af vores projekt vi har testet hvis vi f.eks. tager udgangspunkt i klassen Track så viste den på et tidspunkt at vi har testet 50\% af klassen Track. Man kunne også dykke længere ned i klassen for at se hvilke specefik funktioner der mangler at testes. 


